%!TEX program = xelatex
%!TEX TS-program = xelatex
%!TEX encoding = UTF-8 Unicode
\documentclass[UTF8,11pt]{article}


\usepackage{latexsym}
\usepackage[empty]{fullpage}
\usepackage{titlesec}
\usepackage{marvosym}
\usepackage[usenames,dvipsnames]{color}
\usepackage{verbatim}
\usepackage{enumitem}
\usepackage[pdftex]{hyperref}
\usepackage{fancyhdr}


\usepackage{url} %这个我也不多说了
\usepackage{fontspec,xltxtra,xunicode} %最新的mactex都有

\defaultfontfeatures{Mapping=tex-text}
\setromanfont{Heiti SC} %设置中文字体
\XeTeXlinebreaklocale “zh”
\XeTeXlinebreakskip = 0pt plus 1pt minus 0.1pt %文章内中文自动换行,可以自行调节



\pagestyle{fancy}
\fancyhf{} % clear all header and footer fields
\fancyfoot{}
\renewcommand{\headrulewidth}{0pt}
\renewcommand{\footrulewidth}{0pt}

% Adjust margins
\addtolength{\oddsidemargin}{-0.525in}
\addtolength{\evensidemargin}{-0.675in}
\addtolength{\textwidth}{1in}
\addtolength{\topmargin}{-.6in}
\addtolength{\textheight}{1.0in}

\urlstyle{same}

\raggedbottom
\raggedright
\setlength{\tabcolsep}{0in}

% Sections formatting
\titleformat{\section}{
  \vspace{-10pt}\scshape\raggedright\large
}{}{0em}{}[\color{black}\titlerule \vspace{-8pt}]

%-------------------------
% Custom commands
\newcommand{\resumeItem}[2]{
  \item\small{
    \textbf{#1}{ #2 \vspace{0pt}}
  }
}

\newcommand{\resumeSubheading}[4]{
  \vspace{0pt}\item
    \begin{tabular*}{0.97\textwidth}{l@{\extracolsep{\fill}}r}
      \textbf{#1} & #2 \\
      \textit{\small#3} & \textit{\small #4} \\
    \end{tabular*}\vspace{-7pt}
}

\newcommand{\resumeSubSecheading}[2]{
  \vspace{0pt}\item
    \begin{tabular*}{0.97\textwidth}{l@{\extracolsep{\fill}}r}
      \textbf{#1} & #2 \\
    \end{tabular*}\vspace{-7pt}
}

\newcommand{\resumeSubItem}[2]{\resumeItem{#1}{#2}\vspace{-4pt}}

\renewcommand{\labelitemii}{$\circ$}

\newcommand{\resumeSubHeadingListStart}{\begin{itemize}[leftmargin=*]}
\newcommand{\resumeSubHeadingListEnd}{\end{itemize}}\vspace{-12pt}}
\newcommand{\resumeItemListStart}{\begin{itemize}[fullwidth]}
\newcommand{\resumeItemListEnd}{\end{itemize}\vspace{-8pt}}

%-------------------------------------------
%%%%%%  CV STARTS HERE  %%%%%%%%%%%%%%%%%%%%%%%%%%%%


\begin{document}

%----------HEADING-----------------
\begin{tabular*}{\textwidth}{l@{\extracolsep{\fill}}r}
  \textbf{\href{http://aljun.me/}{\Large 陈阳洋}} & Email : \href{mailto:gagasalamer@outlook.com}{gagasalamer@outlook.com}\\
  \href{http://aljun.me/}{Tech博客: http://www.aljun.me} & 手机 : (86) 183-1057-6552 \\
  & Github: \href{https://github.com/salamer}{github.com/salamer}
\end{tabular*}\vspace{-10pt}


%-----------EDUCATION-----------------
\section{教育背景}
  \resumeSubHeadingListStart
    \resumeSubheading
      {北京化工大学}{北京}
      {理学院 / 应用化学}{2014.9  -- 2018.7}
  \resumeSubHeadingListEnd


%-----------EXPERIENCE-----------------
\section{实践经历}
  \resumeSubHeadingListStart
    \resumeSubheading
      {\href{http://bytedance.com/}{字节跳动}}{北京}
      {后端工程师}{2018.6 -- 至今}
      \resumeItemListStart
        \resumeItem{}
          {负责字节跳动旗下最大垂直业务懂车帝的基础+直播两大业务线的研发工作(包含了所有Feed, 直播, 图文视频详情页, 私信, 消息, 资源位等服务, 覆盖懂车帝App 75\%以上流量及头条系部分汽车相关模块的流量)}
        \resumeItem{}
          {主导懂车帝服务各级别容灾降级的设计落地, 做到在网关层(nginx)稳定的情况下App的基本形态与功能稳定, 在字节系春节等各种导量的流量激增活动中保持各服务稳定可用}
        \resumeItem{}
          {优化feed分发服务与重构频道分发服务, 使用户频道个性化, 丰富关注, 精选等频道, 优化feed召回架构与推荐消重结构, 丰富feed元素, 提升feed效果, 整体改进了feed产品架构, 显著提升分发效率}
        \resumeItem{}
          {主导资源位投放服务(包括功能小球, banner, 弹窗等)的设计落地, 采用规则引擎应对各场景定向需求, 开发独立的Protobuf Generator应对各场景不同的数据结构, 框架限制, 并满足其向后扩展性, 对投放素材及规则采用工单制, 并增加测试状态, 保证投放操作安全性}
        \resumeItem{}
          {主导定向站内信投放服务的设计落地, 采用图数据库等实现用户定向召回, 利用消息队列削峰, 并在每一阶段设计限流降级保护下游安全稳定, 实现百万级站内消息分钟级别投放}
      \resumeItemListEnd
    \resumeSubheading
      {\href{http://baidu.com/}{百度 (人工智能部 知识图谱组)}}{北京}
      {算法实习生}{2017.5 -- 2017.9}
      \resumeItemListStart
        \resumeItem{}
          {在Hadoop平台上使用朴素贝叶斯方法对已有数据进行数据挖掘,新增300万的知识图谱连接数据,正确率在95\%\+}
        \resumeItem{}
          {使用内部深度学习框架LegoNet应用神经网络对搜索的点击结果的数据进行处理, 优化现有搜索结果的排序}
      \resumeItemListEnd
    \resumeSubheading
      {\href{http://megvii.com}{旷视科技 (Face++)}}{北京}
      {软件开发实习生}{2016.6 -- 2016.10}
      \resumeItemListStart
        \resumeItem{}
          {主导基于Protobuf IDL的网关生成器. 基于Protobuf Generator, 生成完整RESTful-RPC的api gateway}
        \resumeItem{}
          {使用D3.js编写了内部深度学习训练的监控, 使用React开发内部云的控制台界面}
      \resumeItemListEnd
    \resumeSubheading
      {\href{https://easyeasyoversea.com}{EasyEasyOversea (https://easyeasyoversea.com)}}{}
      {独立开发}{2017.1 -- 2017.2}
      \resumeItemListStart
        \resumeItem{}
          {最好的留学工具, 改变了当下学生想留学但信息不对称的问题, 简化了申请海外院校和企业的过程, 独立设计开发运营, 刚上线一周, 搜索次数就超十万+, 被三家天使投资主动联系, 被六家该领域主要公司主动寻求合作}
          \resumeItem{}
          {应用技术包括ElasticSearch, Redis, Django, 爬虫及其调度, 前端包括React, D3.js, Material Design}  
      \resumeItemListEnd
    \resumeSubheading
      {\href{http://www.ict.ac.cn/}{中国科学院计算研究所 (计算机体系结构国家重点实验室)} }{北京}
      {实习生}{2016.12 -- 2017.1}
      \resumeItemListStart
        \resumeItem{}
          {在孙毓忠老师组, 负责组内某大数据平台项目(非公开)开发工作, 期间了解部分各大数据平台原理}
      \resumeItemListEnd
    % \resumeSubheading
    %   {\href{https://github.com/salamer/Juicy}{Juicy (github.com/salamer/Juicy)} }{}
    %   {独立开发}{2017.12 -- 2018.1}
    %   \resumeItemListStart
    %     \resumeItem{}
    %       {分布式 K/V 数据库,其中自编写了分布式一致性协议Raft,还实现了数据压缩和持久化功能}
    %   \resumeItemListEnd

  \resumeSubHeadingListEnd

\section{比赛经历}
  \resumeSubHeadingListStart
    \resumeSubSecheading
      {美国数学建模大赛(MCM/ICM)(队长) / 二等奖(Honorable Mention)}{2016.2}
      \resumeItemListStart
        \resumeItem{}
          {巧用Kmeans方法对取水点建模, 并基于灰色系统理论进行建模, 对阿尔及利亚的可用水资源,医疗投入进行模型优化}
      \resumeItemListEnd
    \resumeSubSecheading
      {亚马逊AWS Hackathon Beijing(队长) / 三等奖}{2016.9}
      \resumeItemListStart
        \resumeItem{}
          {应用Golang + Javascript开发, 使用了Grpc, React, Docker技术 基于JSON 配置的sensor监控服务}
      \resumeItemListEnd
    \resumeSubSecheading
      {BloomBerg Code Your Way to Global Data / 一等奖}{2016.11}
      \resumeItemListStart
        \resumeItem{}
          {编写脚步自动化下载并购案例,并使用Word2Vec算法找出相关度很高的词进行自动化文本抓取与处理}
      \resumeItemListEnd
    \resumeSubHeadingListEnd
%-----------PROJECTS-----------------
\section{获奖}
  \resumeSubHeadingListStart
    \resumeSubSecheading
      {人民三等奖学金}{2015.7}
  \resumeSubHeadingListEnd
%
%--------PROGRAMMING SKILLS------------
\section{技能}
  \resumeSubHeadingListStart
    \resumeSubItem{对流量分发型业务熟悉, 有丰富经验}
      {}
    \resumeSubItem{熟悉使用Go, 掌握Python, 了解 C, Javascript, Ruby, Java}
      {}
    \resumeSubItem{掌握基本数据结构和算法, 常见数据挖掘, 机器学习相关基本原理及应用}
      {}
    \resumeSubItem{熟悉分布式原理, Raft协议, 数据库原理}
      {}
    \resumeSubItem{语言: 英语(CET-6), 日语(N2)}
      {}
  \resumeSubHeadingListEnd


%-------------------------------------------
\end{document}
