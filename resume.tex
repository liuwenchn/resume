%!TEX program = xelatex
%!TEX TS-program = xelatex
%!TEX encoding = UTF-8 Unicode
\documentclass[UTF8,11pt]{article}


\usepackage{latexsym}
\usepackage[empty]{fullpage}
\usepackage{titlesec}
\usepackage{marvosym}
\usepackage[usenames,dvipsnames]{color}
\usepackage{verbatim}
\usepackage{enumitem}
\usepackage[pdftex]{hyperref}
\usepackage{fancyhdr}


\usepackage{url} %这个我也不多说了
\usepackage{fontspec,xltxtra,xunicode} %最新的mactex都有

\defaultfontfeatures{Mapping=tex-text}
\setromanfont{Heiti SC} %设置中文字体
\XeTeXlinebreaklocale “zh”
\XeTeXlinebreakskip = 0pt plus 1pt minus 0.1pt %文章内中文自动换行,可以自行调节



\pagestyle{fancy}
\fancyhf{} % clear all header and footer fields
\fancyfoot{}
\renewcommand{\headrulewidth}{0pt}
\renewcommand{\footrulewidth}{0pt}

% Adjust margins
\addtolength{\oddsidemargin}{-0.425in}
\addtolength{\evensidemargin}{-0.875in}
\addtolength{\textwidth}{1in}
\addtolength{\topmargin}{-.7in}
\addtolength{\textheight}{1.0in}

\urlstyle{same}

\raggedbottom
\raggedright
\setlength{\tabcolsep}{0in}

% Sections formatting
\titleformat{\section}{
  \vspace{-10pt}\scshape\raggedright\large
}{}{0em}{}[\color{black}\titlerule \vspace{-8pt}]

%-------------------------
% Custom commands
\newcommand{\resumeItem}[2]{
  \item\small{
    \textbf{#1}{ #2 \vspace{0pt}}
  }
}

\newcommand{\resumeSubheading}[4]{
  \vspace{0pt}\item
    \begin{tabular*}{0.97\textwidth}{l@{\extracolsep{\fill}}r}
      \textbf{#1} & #2 \\
      \textit{\small#3} & \textit{\small #4} \\
    \end{tabular*}\vspace{-7pt}
}

\newcommand{\resumeSubSecheading}[2]{
  \vspace{0pt}\item
    \begin{tabular*}{0.97\textwidth}{l@{\extracolsep{\fill}}r}
      \textbf{#1} & #2 \\
    \end{tabular*}\vspace{-7pt}
}

\newcommand{\resumeSubItem}[2]{\resumeItem{#1}{#2}\vspace{-4pt}}

\renewcommand{\labelitemii}{$\circ$}

\newcommand{\resumeSubHeadingListStart}{\begin{itemize}[leftmargin=*]}
\newcommand{\resumeSubHeadingListEnd}{\end{itemize}}\vspace{-12pt}}
\newcommand{\resumeItemListStart}{\begin{itemize}[fullwidth]}
\newcommand{\resumeItemListEnd}{\end{itemize}\vspace{-8pt}}

%-------------------------------------------
%%%%%%  CV STARTS HERE  %%%%%%%%%%%%%%%%%%%%%%%%%%%%


\begin{document}

%----------HEADING-----------------
\begin{tabular*}{\textwidth}{l@{\extracolsep{\fill}}r}
  \textbf{\href{http://aljun.me/}{\Large 陈阳洋}} & Email : \href{mailto:salamer\_gaga@163.com}{salamer\_gaga@163.com}\\
  \href{http://aljun.me/}{Tech博客: http://www.aljun.me} & 手机 : (86) 183-1057-6552 \\
  & Github: \href{https://github.com/salamer}{github.com/salamer}
\end{tabular*}\vspace{-10pt}


%-----------EDUCATION-----------------
\section{教育背景}
  \resumeSubHeadingListStart
    \resumeSubheading
      {北京化工大学}{北京}
      {理学院 / 应用化学}{2014.9  -- 2018.7}
  \resumeSubHeadingListEnd


%-----------EXPERIENCE-----------------
\section{实践经历}
  \resumeSubHeadingListStart
    \resumeSubheading
      {\href{http://baidu.com/}{百度}}{北京}
      {算法实习生}{2017.5 -- 2017.8}
      \resumeItemListStart
        \resumeItem{}
          {利用内部Hadoop平台使用朴素贝叶斯方法对百度百科已有数据进行数据挖掘,新增300万未知的知识图谱数据,正确率在95\%以上}
        \resumeItem{}
          {使用内部深度学习框架LegoNet,用CNN对搜索的点击结果的数据进行处理,优化现有搜索结果的排序}
      \resumeItemListEnd
    \resumeSubheading
      {\href{http://megvii.com}{Megvii(Face++)}}{北京}
      {软件开发实习生}{2016.6 -- 2016.10}
      \resumeItemListStart
        \resumeItem{}
          {使用Golang编写了基于Google ProtoBuf的JSON-gRPC的网关生成器,可提取一份配置好的proto文件,生成前端提供RESTful服务后端走RPC的服务}
        \resumeItem{}
          {使用D3.js编写了内部深度学习训练的监控, 通过编写前端缓存结构, 实现数据动态加载, 优化海量数据的前端展示问题}
        \resumeItem{}
          {使用React写内部云的控制台节目}
      \resumeItemListEnd
    \resumeSubheading
      {\href{https://easyeasyoversea.com}{EasyEasyOversea (https://easyeasyoversea.com)}}{}
      {独立开发}{2017.1 -- 2017.2}
      \resumeItemListStart
        \resumeItem{}
          {最好的留学工具, 改变了当下学生想留学但信息不对称的问题, 简化了申请海外院校和企业的过程, 独立设计开发运营}
          \resumeItem{}
          {刚上线一周, 搜索次数就超十万+, 被三家天使投资主动联系, 被六家该领域主要公司主动寻求合作}
          \resumeItem{}
          {基于MongoDB和Gevent编写了一套高性能搜索爬虫组件, 使用ElasticSearch作为搜索后端, 用Golang写了搜索缓存结构来减少硬盘IO并发, 使用Django Rest并基于Mysql写了应用的论坛, 并使用Redis优化并发, 并使用Celery编写了一些需要等待的异步组件, 使用React编写并设计了的Material Design的前端, 并使用D3.js做数据展示}        
      \resumeItemListEnd

    \resumeSubheading
      {\href{http://zhaduixueshe.com/}{扎堆学社官网 (http://zhaduixueshe.com)} }{}
      {独立开发}{2015.7 -- 2015.9}
      \resumeItemListStart
        \resumeItem{}
          {校内的学霸复习资料分享平台, 实现了组织博客系统, 学霸资料分科目分享平台, 对所有文章的Markdown支持, 实现了后台功能, 全面的在线报名系统, 通过访问者的角色, 控制权限管理, 实现明确的权限管理与分工}
        \resumeItem{}
          {基于Django框架搭建的后端, 使用MySql作为数据库, 利用Nginx 实现静态文件的负载均衡, 前端基于Bootstrap与JQeury,高峰期日访问量达到3000+}
        
      \resumeItemListEnd
%    \resumeSubheading
  %    {\href{https://github.com/salamer/Juicy}{Juicy (github.com/salamer/Juicy)} }{}
    %  {独立开发}{2017.12 -- 2018.1}
      %\resumeItemListStart
        %\resumeItem{}
          %{分布式 K/V 数据库,其中自编写了分布式一致性协议Raft}
      %\resumeItemListEnd

  \resumeSubHeadingListEnd

\section{比赛经历}
  \resumeSubHeadingListStart
    \resumeSubSecheading
      {美国数学建模大赛(MCM/ICM)(队长) / 二等奖(Honorable Mention)}{2016.2}
      \resumeItemListStart
        \resumeItem{}
          {对阿尔及利亚的可用水资源,医疗投入,政府监管等数据进行拟合分析得相应的数据关系,并基于灰色系统理论进行建模}
        \resumeItem{}
          {巧用KNN方法对取水点建模}
      \resumeItemListEnd
    \resumeSubSecheading
      {亚马逊AWS Hackathon Beijing(队长) / 三等奖}{2016.9}
      \resumeItemListStart
        \resumeItem{}
          {使用Golang + Javascript开发, 使用了Grpc, React, Docker技术, 可基于一份JSON 配置文件, 生成一整套前端监视器服务, 数据来源是通过Rpc调用的分散的集群的数据,并对监视器的信息源提高结构支持}
      \resumeItemListEnd
    \resumeSubSecheading
      {BloomBerg Code Your Way to Global Data / 一等奖}{2016.11}
      \resumeItemListStart
        \resumeItem{}
          {编写脚步自动化下载并购案例,并使用Word2Vec算法找出相关度很高的词进行自动化文本抓取与处理}
      \resumeItemListEnd
    \resumeSubHeadingListEnd
%-----------PROJECTS-----------------
\section{获奖}
  \resumeSubHeadingListStart
    \resumeSubSecheading
      {人民三等奖学金}{2015.7}
  \resumeSubHeadingListEnd
%
%--------PROGRAMMING SKILLS------------
\section{技能}
  \resumeSubHeadingListStart
    \resumeSubItem{熟悉Web前后端开发,数据处理与抓取, HTTP协议}
      {}
    \resumeSubItem{熟悉使用Python, Go, C, Javascript, R, Ruby, CSS}
      {}
    \resumeSubItem{掌握基本数据结构和算法,常见数据挖掘、机器学习相关基本原理及应用}
      {}
    \resumeSubItem{熟悉Linux基本操作,数据库操作}
      {}
    \resumeSubItem{语言: 英语(CET-6) , 日语(基本日常交流)}
      {}
  \resumeSubHeadingListEnd


%-------------------------------------------
\end{document}
